\documentclass{exam}

\usepackage{amsfonts}
\usepackage{amssymb}
\usepackage{mathtools}
\usepackage{braket}
\usepackage{forloop}
\usepackage{amsthm}
\usepackage[backend=biber]{biblatex}

\theoremstyle{plain}
\newtheorem{assumption}{Assumption}

\theoremstyle{definition}
\newtheorem{definition}{Definition}

\newtheorem{lemma}{Lemma}

\input{preamble.sty}

\addbibresource{references.bib}

\begin{document}

\title{MAT258S25 Proof 1}
\author{Kishan S Patel}
\maketitle

\renewcommand{\qedsymbol}{QED}

\noindent\rule{\textwidth}{1pt}

\begin{definition}
	Any element in the set $\Q$ can be expressed as $\frac{q}{p}$,
	where $p,q \in \Z$, $q \neq 0$

	$$P_{\Q} = \forall \pa{p \in \Z, q \in {\Z \setminus \set{0} }} \ba{ \frac{p}{q} \in \Q }$$
\end{definition}


\begin{assumption}
	The set $\Z$ is closed under addition.
	$$P^{+}=\forall x, y \in \Z \ba { (x+y) \in \Z }$$
\end{assumption}

\begin{assumption}
	The set $\Z$ is closed under multiplication.
	$$P^{\times} = \forall x, y \in \Z \ba { xy \in \Z }$$
\end{assumption}

\noindent\rule{\textwidth}{1pt}

\begin{questions}
	\question For any given $r, s \in \mathbb{Q}$, prove that $\pa{r+s} \in \mathbb Q$.

\begin{proof}
	\begin{gather*}
		\llet r = \frac{a}{b}, s = \frac{c}{d} \\
		r \in \Q \implies a \in \Z, b \in \Z\setminus\set{0}\\
		s \in \Q \implies c \in \Z, d \in \Z\setminus\set{0}
	\end{gather*}

	\begin{align*}
		r+s & =  \frac{a}{b} + \frac{c}{d}        \\
		    & = \frac{a d}{b d} + \frac{c b}{d b} \\
		    & = \frac{a d + c b}{b d}
	\end{align*}
	\begin{align*}
		P^{\times}            & \implies  {a d, c b, b d} \in \Z     \\
		P^{\times} \wedge P^+ & \implies \pa{ a d + c b } \in \Z     \\
		0 \not\in \set{b, d}  & \implies		b d \in \Z\setminus\set{0}
	\end{align*}

	\begin{equation*}
		\begin{split}
			\pa{a d  +    c b} \in \Z           & \\
			b d  \in \pa{\Z \setminus \set{0} } &
		\end{split}
		\implies  \frac{a d + c b}{b d} \in \Q  \rightarrow	 r+s \in \Q
	\end{equation*}

\end{proof}

	\question $\llet r \in \Q, \llet s \in \R\setminus\Q$, prove that $r+s \in \R\setminus\Q$

\begin{proof}
	Assume the the proposition $P_2$ to be false, $\forall r \in \Q, s \in \R\setminus\Q \,\ba{r+s \in \R\setminus\Q}$

	\begin{gather*}
		\llet r  = \frac{p}{q} \\
		\llet r+s = \frac{t}{v} \\
		\begin{split}
			P_{\Q}            & \implies p \in \Z, q \in \Z\setminus\set{0}  \\
			P_2 \wedge P_{\Q} & \implies  t \in \Z, v \in \Z\setminus\set{0}
		\end{split}
	\end{gather*}

	\begin{align*}
		r + s             & = \frac{t}{v}               \\
		\frac{p}{q} + s   & = \frac{t}{v}               \\
		\frac{vp}{q} + vs & = t                         \\
		vp + vsq          & = tq                        \\
		s                 & = \frac{tq-vp}{vq}          \\
		                  & = \frac{tq + (-1)(vp) }{vq}
	\end{align*}

	\begin{align*}
		\ba{q,v \in \Z \setminus \set{0} } \wedge P^{\times} & \implies vq \in \Z \setminus \set{0}   \\
		\ba{p,q\in \Z} \wedge P^{\times}                     & \implies \set{tq, vp} \subset \Z       \\
		P^{+} \wedge P^{+}                                   & \implies tq+(-1)vp \in \Z              \\
		\therefore\,                                         & \frac{tq+(-1)vp}{vq} \in \R\setminus\Q
		\\ \rightarrow\,& s \in \R\setminus\Q
	\end{align*}

	This provides a contradiction regarding $s\in\R\setminus\Q$ , as $\neg P_2 \implies s \in \Q$
\end{proof}

\end{questions}

\end{document}
