\documentclass{exam}

\usepackage{amsfonts}
\usepackage{amssymb}
\usepackage{mathtools}
\usepackage{braket}
\usepackage{forloop}
\usepackage{amsthm}
\usepackage[backend=biber]{biblatex}

\theoremstyle{plain}
\newtheorem{assumption}{Assumption}

\theoremstyle{definition}
\newtheorem{definition}{Definition}

\newtheorem{lemma}{Lemma}


% huge align
\newcommand{\ha}[1]{{\huge{\begin{align*}#1\end{align*}}}}



% P & S Are excluded 
% \newcommand{\A}[0]{{\mathbb A}}
% \newcommand{\B}[0]{{\mathbb B}}
% \newcommand{\C}[0]{{\mathbb C}}
% \newcommand{\D}[0]{{\mathbb D}} 
% \newcommand{\E}[0]{{\mathbb E}} 
% \newcommand{\F}[0]{{\mathbb F}} 
% \newcommand{\G}[0]{{\mathbb G}} 
% \newcommand{\H}[0]{{\mathbb H}} 
% \newcommand{\I}[0]{{\mathbb I}} 
% \newcommand{\J}[0]{{\mathbb J}} 
% \newcommand{\K}[0]{{\mathbb K}} 
% \newcommand{\L}[0]{{\mathbb L}} 
% \newcommand{\M}[0]{{\mathbb M}} 
\newcommand{\N}[0]{{\mathbb N}}
% \newcommand{\O}[0]{{\mathbb O}} 
% \newcommand{\P}[0]{{\mathbb P}} 
\newcommand{\Q}[0]{{\mathbb Q}}
\newcommand{\R}[0]{{\mathbb R}}
% \renewcommand{\S}[0]{{\mathbb S}}
% \newcommand{\T}[0]{{\mathbb T}}
% \newcommand{\U}[0]{{\mathbb U}}
% \newcommand{\V}[0]{{\mathbb V}}
% \newcommand{\W}[0]{{\mathbb W}}
% \newcommand{\X}[0]{{\mathbb X}}
% \newcommand{\Y}[0]{{\mathbb Y}}
\newcommand{\Z}[0]{{\mathbb Z}}

% Calculus
% \newcommand{\d}[0]{{\mathrm{d}}}
\newcommand{\deriv}[2]{ \frac{ \d{#1} }{ \d{#2} } }
\newcommand{\pderiv}[2]{ \frac{ \partial{#1} }{ \partial{#2} } }

\newcommand{\nderiv}[3]{ \frac{ \d^{#1}{#2} }{ \d{#3}^{#1} } }
\newcommand{\npderiv}[3]{ \frac{ \partial^{#1}{#2} }{ \partial{#3}^{#1} } }

% Linear Algebra 
\renewcommand{\vector}[1]{ \overrightarrow{#1} }
\newcommand{\vecn}[1]{ {\hat #1} }

\newcommand{\mat}[1]{{ \begin{bmatrix} #1 \end{bmatrix} }}
\newcommand{\mats}[1]{{ \ba{\begin{smallmatrix} #1 \end{smallmatrix}} }}

\newcommand{\pmat}[1]{{ \begin{pmatrix} #1 \end{pmatrix} }}
\newcommand{\pmats}[1]{{ \pa{\begin{smallmatrix} #1 \end{smallmatrix}} }}

\newcommand{\emat}[1]{{ \begin{ematrix} #1 \end{ematrix} }}
\newcommand{\emats}[1]{{ \begin{smallmatrix} #1 \end{smallmatrix} }}

\newcommand{\vmat}[1]{{ \begin{vmatrix} #1 \end{vmatrix} }}

\newcommand{\rowechelon}[1]{{
			\left[\begin{array}{ccc|c} #1 \end{array}\right]
		}}

\newcommand{\augmented}[2]{{
			\left[\begin{array}{#1} #2 \end{array}\right]
		}}

% Generic Notatino
\newcommand{\paren}[1]{{ \left(#1\right) }}

\newcommand{\pa}[1]{{ \left(#1\right) }}
\newcommand{\ba}[1]{{ \left[#1\right] }}


\newcommand{\llet}[0]{ {\text{let } } }
\newcommand{\undefined}[0]{ {\text{undefined.} } }

\newcommand{\op}[1]{ {\operatorname{#1} } }

\newcommand{\brt}[2]{ {\root {#1} \of {#2} } }

\newcommand{\proj}[1]{ { \op{proj}_{#1} }}
\newcommand{\projperp}[1]{ { \op{proj}_{#1\perp} } }

\newcommand{\norm}[1]{{ {\left\lVert #1 \right\rVert} }}
\newcommand{\norms}[1]{{ {\lVert #1 \rVert} }}

% CS 
\newcommand{\hex}[1]{{ \pa{\mathrm{#1}}_{16} }}
\newcommand{\bin}[1]{{ \pa{#1}_{2} }}
\newcommand{\binb}[2]{{ \pa{#1}^{#2}_{2} }}
\newcommand{\dec}[1]{{ \pa{#1}_{10} }}


\newcommand{\true}[0]{{ \mathrm{true} }}
\newcommand{\false}[0]{{ \mathrm{false} }}

\renewcommand{\ba}[1]{{ \left[ {#1} \right] }}

\newcommand{\ceil}[1]{{ \left\lceil {#1} \right\rceil }}
\newcommand{\floor}[1]{{ \left\lfloor {#1} \right\rfloor }}

\newcommand{\ang}[1]{{ \left\langle {#1} \right\rangle }}

\newcommand{\transpose}[1]{ { {#1}^{\intercal} } }





\addbibresource{references.bib}

\begin{document}

\title{MAT258S25 Proof 1}
\author{Kishan S Patel}
\maketitle

\noindent\rule{\textwidth}{1pt}

\begin{definition}
	Any element in the set $\Q$ can be expressed as $\frac{q}{p}$,
	where $p,q \in \Z$, $q \neq 0$

	$$P_{\Q} = \forall \pa{p \in \Z, q \in {\Z \setminus \set{0} }} \ba{ \frac{p}{q} \in \Q }$$
\end{definition}

\begin{assumption}
	The set $\Z$ is closed under addition.
	$$P^{+}=\forall x, y \in \Z \ba { (x+y) \in \Z }$$
\end{assumption}

\begin{assumption}
	The set $\Z$ is closed under multiplication.
	$$P^{\times} = \forall x, y \in \Z \ba { xy \in \Z }$$
\end{assumption}

\noindent\rule{\textwidth}{1pt}

\begin{questions}
	\question For any given $r, s \in \mathbb{Q}$, prove that $\pa{r+s} \in \mathbb Q$.

\begin{proof}
	\begin{gather*}
		\llet r = \frac{a}{b}, s = \frac{c}{d} \\
		r \in \Q \implies a \in \Z, b \in \Z\setminus\set{0}\\
		s \in \Q \implies c \in \Z, d \in \Z\setminus\set{0}
	\end{gather*}

	\begin{align*}
		r+s & =  \frac{a}{b} + \frac{c}{d}        \\
		    & = \frac{a d}{b d} + \frac{c b}{d b} \\
		    & = \frac{a d + c b}{b d}
	\end{align*}
	\begin{align*}
		P^{\times}            & \implies  {a d, c b, b d} \in \Z     \\
		P^{\times} \wedge P^+ & \implies \pa{ a d + c b } \in \Z     \\
		0 \not\in \set{b, d}  & \implies		b d \in \Z\setminus\set{0}
	\end{align*}

	\begin{equation*}
		\begin{split}
			\pa{a d  +    c b} \in \Z           & \\
			b d  \in \pa{\Z \setminus \set{0} } &
		\end{split}
		\implies  \frac{a d + c b}{b d} \in \Q  \rightarrow	 r+s \in \Q
	\end{equation*}

\end{proof}

	\question $\llet r \in \Q, \llet s \in \R\setminus\Q$, prove that $r+s \in \R\setminus\Q$

\begin{proof}
	Assume the the proposition $P_2$ to be false, $\forall r \in \Q, s \in \R\setminus\Q \,\ba{r+s \in \R\setminus\Q}$

	\begin{gather*}
		\llet r  = \frac{p}{q} \\
		\llet r+s = \frac{t}{v} \\
		\begin{split}
			P_{\Q}            & \implies p \in \Z, q \in \Z\setminus\set{0}  \\
			P_2 \wedge P_{\Q} & \implies  t \in \Z, v \in \Z\setminus\set{0}
		\end{split}
	\end{gather*}

	\begin{align*}
		r + s             & = \frac{t}{v}               \\
		\frac{p}{q} + s   & = \frac{t}{v}               \\
		\frac{vp}{q} + vs & = t                         \\
		vp + vsq          & = tq                        \\
		s                 & = \frac{tq-vp}{vq}          \\
		                  & = \frac{tq + (-1)(vp) }{vq}
	\end{align*}

	\begin{align*}
		\ba{q,v \in \Z \setminus \set{0} } \wedge P^{\times} & \implies vq \in \Z \setminus \set{0}   \\
		\ba{p,q\in \Z} \wedge P^{\times}                     & \implies \set{tq, vp} \subset \Z       \\
		P^{+} \wedge P^{+}                                   & \implies tq+(-1)vp \in \Z              \\
		\therefore\,                                         & \frac{tq+(-1)vp}{vq} \in \R\setminus\Q
		\\ \rightarrow\,& s \in \R\setminus\Q
	\end{align*}

	This provides a contradiction regarding $s\in\R\setminus\Q$ , as $\neg P_2 \implies s \in \Q$
\end{proof}

\end{questions}

\end{document}
